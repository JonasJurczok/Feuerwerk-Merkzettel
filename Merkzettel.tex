\documentclass[10pt,landscape]{article}
\usepackage{multicol}
\usepackage{calc}
\usepackage{ifthen}
\usepackage[landscape]{geometry}
\usepackage{amsmath,amsthm,amsfonts,amssymb}
\usepackage{color,graphicx,overpic}
\usepackage{hyperref}


\pdfinfo{
  /Title (Feuerwerk Merkzettel.pdf)
  /Creator (TeX)
  /Producer (pdfTeX 1.40.0)
  /Author (Jonas Jurczok)
  /Subject (Merkzettel zum Umgang mit Kat 3 Feuerwerk)
  /Keywords (Feuerwerk, Merkzettel, Kategorie 3, Kat 3)}

% This sets page margins to .5 inch if using letter paper, and to 1cm
% if using A4 paper. (This probably isn't strictly necessary.)
% If using another size paper, use default 1cm margins.
\ifthenelse{\lengthtest { \paperwidth = 11in}}
    { \geometry{top=.5in,left=.5in,right=.5in,bottom=.5in} }
    {\ifthenelse{ \lengthtest{ \paperwidth = 297mm}}
        {\geometry{top=1cm,left=1cm,right=1cm,bottom=1cm} }
        {\geometry{top=1cm,left=1cm,right=1cm,bottom=1cm} }
    }

% Turn off header and footer
\pagestyle{empty}

% Redefine section commands to use less space
\makeatletter
\renewcommand{\section}{\@startsection{section}{1}{0mm}%
                                {-1ex plus -.5ex minus -.2ex}%
                                {0.5ex plus .2ex}%x
                                {\normalfont\large\bfseries}}
\renewcommand{\subsection}{\@startsection{subsection}{2}{0mm}%
                                {-1explus -.5ex minus -.2ex}%
                                {0.5ex plus .2ex}%
                                {\normalfont\normalsize\bfseries}}
\renewcommand{\subsubsection}{\@startsection{subsubsection}{3}{0mm}%
                                {-1ex plus -.5ex minus -.2ex}%
                                {1ex plus .2ex}%
                                {\normalfont\small\bfseries}}
\makeatother


% Don't print section numbers
\setcounter{secnumdepth}{0}


\setlength{\parindent}{0pt}
\setlength{\parskip}{0pt plus 0.5ex}

%My Environments
\newtheorem{example}[section]{Example}
% -----------------------------------------------------------------------

\begin{document}
\raggedright
\footnotesize
\begin{multicols}{3}


% multicol parameters
% These lengths are set only within the two main columns
%\setlength{\columnseprule}{0.25pt}
\setlength{\premulticols}{1pt}
\setlength{\postmulticols}{1pt}
\setlength{\multicolsep}{1pt}
\setlength{\columnsep}{2pt}

\begin{center}
     \Large{\underline{Merkzettel Kategorie 3 Feuerwerk}} \\
\end{center}

\section{Begriffe}
\subsection{Kategorien}
Feuerwerk wird in vier Kategorien unterschieden \cite{Kategorien}
\begin{enumerate}
\item Kategorie 1 
Tischfeuerwerk, Knallerbsenm oder Wunderkerzen
\item Kategorie 2
Silvesterfeuerwerkt (Batterien, Raketen, Böller)
\item Kategorie 3
Feuerwerk mittelgroßer Gefahr. Ebenfalls Batterien, Raketen und Böller. Benötigt eine Erlaubnis nach \$27SprenGG (hier Schein genannt).
\item Kategorie 4
Profi/Großfeuerwerk. Nur benutzbar mit spezieller Ausbildung.
\end{enumerate}

\subsection{Stoffklassen}
Gefahrstoffe werden in Klassen (1-9) unterteilt. Schwarzpulver und andere Explosivstoffe sind in Klasse 1 einsortiert \cite{Stoffklasse}. 

Da Feuerwerk mit Explosivstoffen hergestellt wird, sind die folgenden Gefahrgutklassen auch für Feuerwerk relevant.
\begin{enumerate}
\item 1.1
Massenexplosionsgefährlich - Alles explodiert gleichzeitig
\item 1.2
Teilexplosionsgefährlich - Die Gesamtmenge explodiert in Teilen
\item 1.3
Massenfeuergefährlich - Alles fängt Feuer
\item 1.4
Reaktion ist auf die Verpackung beschränkt
\end{enumerate}

\subsection{Brutto vs Netto}
\begin{enumerate}
\item NEM - Netto Explosivmasse \cite{NEM} \\
Menge an Explosionsstoff. Auf der Verpackung angegeben ("Menge an Schwarzpulver")
\item Bruttomasse \cite{NEM} \\
Gesamtgewicht des Feuerwerks inklusive Pappe, Papier etc.. Auf der Verpackung angegeben.
\end{enumerate}

\section{Kauf}
Grundsätzlich kann jeder Scheininhaber Kategorie 3 Feuerwerk erwerben.
Die Menge ist dabei höchstens im Schein begrenzt.
WICHTIG: Beschränkungen zum Transport beachten!


\subsection{Sachkunde}
Es gibt den Schein mit und ohne Sachkunde.
Sachkunde wird benötigt um bestimmte Artikel der Kategorie 3 und höher zu erwerben (z.B. Zink Raketen)

\section{Transport}
\subsubsection{Mengen}
Für den Transport im privaten PKW \cite{Transport} gelten die folgenden Maximalmengen je nach Stoffklasse: \cite{Transportmenge}
\begin{enumerate}
\item 1.1 - 1.4 - 3 Kg NEM
\item 1.1 - 1.3 - 5 Kg Bruttomasse
\item 1.4 - 50Kg Bruttomasse
\end{enumerate}

\subsubsection{Dauer}
Das Feuerwerk darf maximal für einen Zeitraum von 72 Stunden transportiert werden. Damit sind Feuerwerke am Wochenende abgedeckt.

\subsubsection{Weglänge}
Es ist der direkte Weg zu wählen.

\subsubsection{Schutzmaßnahmen}
Es ist sicherzustellen\cite{Transport}, dass die Ladung 

\begin{enumerate}
\item nicht frei wird (sich in einem Karton oder ähnlichem befindet).
\item mit einer stabilden Decke o.Ä. abgedeckt ist
\item nicht direkter Sonneneinstrahlung ausgesetzt ist.
\end{enumerate}

\section{Aufbewahrung}
\begin{enumerate}
\item Räumlichkeiten
\item Feuerhemmend
\item Abschließbar
\item Blickschutz
\item Warnzeichen
\end{enumerate}


\section{Abbrennen}
Etc.

\section{Rechtliches}
Die Inhalte dieses Merkzettels wurden sorgfältig geprüft und nach bestem Wissen erstellt. Dennoch besteht für die hier aufgeführten Informationen kein Anspruch auf Vollständigkeit, Aktualität, Qualität und Richtigkeit. Es wird keine Verantwortung für Schäden übernommen, die durch das Vertrauen auf die Inhalte dieses Dokumentes entstehen.

% You can even have references
\rule{0.3\linewidth}{0.25pt}
\scriptsize
\bibliographystyle{plain}
\begin{thebibliography}{9}

\bibitem{Kategorien}
http://www.feuerwerk.net/wiki/Kategorie

\bibitem{Stoffklasse}
https://de.wikipedia.org/wiki/Gefahrgutklasse

\bibitem{NEM}
http://www.feuerwerk.net/wiki/Nettoexplosivstoffmasse

\bibitem{Transport}
Anlage zum Abkommen über die internationale Beförderung gefährlicher Güter auf der Straße, Abschnitt 1.1.3.1 a), 
http://gaa.baden-wuerttemberg.de/servlet/is/16496/2\_2\_1\_1.pdf

\bibitem{Transportmenge}
(Gefahrgutverordnung Straße, Eisenbahn und Binnenschifffahrt - GGVSEB), Abschnitt 2.1 a),
http://gaa.baden-wuerttemberg.de/servlet/is/16496/2\_2\_1.pdf


\end{thebibliography}

\end{multicols}
\end{document}